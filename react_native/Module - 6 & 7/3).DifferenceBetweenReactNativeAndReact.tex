difference between React Native and React ?

1).Installation Process	
=> React.js	: ReactJS is a JavaScript library installed via the npm package manager.	
=> React Native : React Native is a command-line interface tool that requires both Node.js and the React Native CLI to be installed.

2).Efficiency	
=> React.js	: ReactJs is more efficient in terms of development time and code reusability.	
=> React Native : React Native is more efficient in terms of performance and memory usage.

3).Technology Base	
=> React.js	: ReactJS is a JavaScript library used for building user interfaces. It uses a Virtual DOM algorithm to enhance performance.	
=> React Native : React Native is a cross-platform mobile development structure or framework that uses components that are native instead of web components as its building blocks. This makes it possible to create apps that have a more native feel and look.

4).Feasibility
=> React.js	: ReactJS is the more feasible option for web application development.	
=> React Native : React Native is the more feasible option for mobile app development.

5).Components	
=> React.js	: ReactJS components are typically written in HTML. These components must be imported into the app before they can be used.	
=> React Native : React Native components are written in JSX, a syntax extension of JavaScript. These components are compiled directly into native code.

6).Compatibility
=> React.js	: It is compatible with a vast range of browsers, including Internet Explorer.	
=> React Native : React Native is a framework used for developing native mobile apps. It is only compatible with the two major mobile platforms, iOS, and Android.

7).Navigation
=> React.js	: ReactJS uses a traditional browser-based approach.	
=> React Native : React Native is a better choice for creating native mobile applications than ReactJS. React Native relies on native platform navigational components.

8).Storage
=> React.js	: ReactJS is a good choice for projects that require high-performance storage, such as dynamic web applications.	
=> React Native : React Native is a better choice for projects that need to be able to scale easily.

9).Rendering	
=> React.js	: Browser code in React is rendered through the Virtual DOM	
=> React Native : React Native uses native APIs to render all components on mobile

