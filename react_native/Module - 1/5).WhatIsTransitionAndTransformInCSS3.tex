What is Transition and Transform in CSS3 ?

The Transform property in CSS moves or modifies the appearance of an element, 
whereas the Transition property seamlessly and gently transitions the element from one state to another.

---------------------------------------------------------------------------------------------------
-> CSS Transform Property

Transform property in CSS is invoked when there is a change in the state of the HTML element.
You can rotate, skew, move and scale elements. It occurs when the state of an element is modified, 
like when you hover the mouse over a button or perform a mouse-click.
We will see how this works in further sections of this blog.

There are three variations of CSS Transform properties in 2D.

transform: TpropertyX(x);
transform: TpropertyY(y);
transform : Tproperty(x,y);
Here Tproperty refers to the element property you want to change, 
x and y can be negative or positive values. CSS Transform property in 3D includes the Z-axis. 
X is the width, Y is the height, and Z gives the depth of the screen.

------------------------------------------------------------------------------------------------------
-> Translate

Translate property changes the position left/right and up/down of the element on the page based 
on the given X (horizontal) and Y (vertical) axes parameters. The positive X-axis parameter 
moves the element to the right, and the negative will do so to the left. The positive Y-axis 
parameter moves the element down, and the positive does so towards up.


----------------------------Exampals----------------------------
#box :hover{ transform: translate(100%,60%);}
#box :hover{ transform: skew(30deg,30deg);
#box :hover{ transform: scale(0.5);}
#box4:hover{ transform: rotate(25deg);}
#box:hover
{
    transform: rotate(120deg) scale(1.5) translateY(-100px);
}